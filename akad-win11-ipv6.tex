%
% Erstellt von Daniel Falkner
% daniel.falkner@akad.de
% 
\documentclass[xcolor=dvipsnames]{beamer}
\usepackage[T1]{fontenc}
\usepackage[utf8]{inputenc}
\usepackage[justification=centering,figurename=Abb.]{caption}

\usetheme{Warsaw}
\usecolortheme[named=OliveGreen]{structure}

\newcommand*{\Title}{Warum ist ein neues Internetprotokoll notwendig?} %Titel
\subtitle{Modul WIN11} %Untertitel
\newcommand*{\Author}{Daniel Falkner + Eugen Grinschuk} %Name
\institute{AKAD Pinneberg + Stuttgart} %Uni
\titlegraphic{\includegraphics[scale=0.2]{akad_logo.png}} %Logo

\title{\Title}
\author{\Author}
\date{\today}

%Pdf Metainformationen
\subject{\Title}
\keywords{}

\begin{document}

%Titelseite
\begin{frame}
    \titlepage
\end{frame}

%Logo auf allen weiteren Folien
%\logo{\includegraphics[scale=0.1]{akad_logo.png}}

%Inhaltsverzeichniss
\frame{\tableofcontents} 

%Ab hier die Folien 
 
\section{Hintergründe}
\begin{frame} %%Eine Folie
 \frametitle{Allgemein}
Das Internet Protocol (IP) ist die Grundlage für die Datenübertragung im Internet. Der sendende Rechner entlässt ein Paket ins Netz. Router, welche die verschiedenen Teilnetze verbinden, sorgen dafür, dass es auf einem (nicht von vornherein festgelegten) Weg durch die Teilnetze den Empfänger erreicht. Sender und Empfänger sind durch eindeutige IP-Adressen identifiziert, die den eigentlichen Nutzdaten vorangestellt werden.
\end{frame}


\subsection{IPv4}
\begin{frame}
 \frametitle{Internet Protocol Version 4}
  \begin{block}{IPv4}
	  \begin{itemize}
	  	\item Definiert 1981 im RFC \footnote{Request for Comments} 791
	  	\item Beispiel Adresse: 8.8.8.8
	  \end{itemize}
  \end{block}
\end{frame}

\subsection{IPv6}
\begin{frame}
  \frametitle{Internet Protocol Version 6}
  \begin{block}{IPv6}
	  \begin{itemize}
  		\item Definiert 1998 im RFC 2460 
  		\item Beispiel Adresse: 2001:4860:4860::8888
	  \end{itemize}
  \end{block}
\end{frame}


%\frame{\tableofcontents[currentsection]}

\section{Nachteile IPv4}
\begin{frame}
  \begin{alertblock}{Nachteile Internet Protocol Version 4}
	  \begin{itemize}
  		\item nur $2^{32}$ (4294967296) = 4 Milliarden Adressen
	  \end{itemize}
  \end{alertblock}
\end{frame}

\section{Vorteile IPv6}
\begin{frame}
  \begin{block}{Vorteile Internet Protocol Version 6}
	  \begin{itemize}
  		\item $2^{128}$ (3,402823669 * $10^{38}$) = > 340 Sextillionen Adressen
	  \end{itemize}
  \end{block}
\end{frame}



\section{Fazit und Ausblick}
\begin{frame}
 \frametitle{Lorem ipsum}
  Lorem ipsum dolor sit amet, consetetur sadipscing elitr, sed diam nonumy eirmod tempor invidunt ut labore et dolore magna aliquyam erat, sed diam voluptua. At vero eos et accusam et justo duo dolores et ea rebum. Stet clita kasd gubergren, no sea takimata sanctus est Lorem ipsum dolor sit amet.
\end{frame}

\section{Quellen}
\begin{frame}
 \frametitle{Quellen}
  \begin{block}{}
	  \begin{itemize}
  		\item IPv4 \url{http://tools.ietf.org/html/rfc791}
  		\item IPv6 \url{http://tools.ietf.org/html/rfc2460}
	  \end{itemize}
  \end{block}
\end{frame}

\subsection*{Ende}
\begin{frame}
	\begin{block}{}	
		\begin{center}
			Vielen Dank für Ihre Aufmerksamkeit. \\
			\Author{}
		\end{center}	
	\end{block}
\end{frame}

\end{document}


