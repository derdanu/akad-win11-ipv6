%
% Erstellt von Daniel Falkner
% daniel.falkner@akad.de
% 
\documentclass[xcolor=dvipsnames]{beamer}
\usepackage[T1]{fontenc}
\usepackage[utf8]{inputenc}
\usepackage[justification=centering,figurename=Abb.]{caption}

\usetheme{Warsaw}
\usecolortheme[named=OliveGreen]{structure}
\renewcommand\thempfootnote{\arabic{mpfootnote}}

\newcommand*{\Title}{Warum ist ein neues Internetprotokoll notwendig?} %Titel
\subtitle{Modul WIN11} %Untertitel
\newcommand*{\Author}{Daniel Falkner + Eugen Grinschuk} %Name
\institute{AKAD Pinneberg + Stuttgart} %Uni
\titlegraphic{\includegraphics[scale=0.2]{akad_logo.png}} %Logo

\title{\Title}
\author{\Author}
\date{\today}

%Pdf Metainformationen
\subject{\Title}
\keywords{}

\begin{document}

%Titelseite
\begin{frame}
    \titlepage
\end{frame}

%Logo auf allen weiteren Folien
%\logo{\includegraphics[scale=0.1]{akad_logo.png}}

%Inhaltsverzeichniss
\frame{\tableofcontents} 

%Ab hier die Folien 
 
\section{Hintergründe}
\begin{frame} %%Eine Folie
 \frametitle{Allgemein}
Das Internet Protocol (IP) ist die Grundlage für die Datenübertragung im Internet. Der sendende Rechner entlässt ein Paket ins Netz. Router, welche die verschiedenen Teilnetze verbinden, sorgen dafür, dass es auf einem (nicht von vornherein festgelegten) Weg durch die Teilnetze den Empfänger erreicht. Sender und Empfänger sind durch eindeutige IP-Adressen identifiziert, die den eigentlichen Nutzdaten vorangestellt werden.
\end{frame}


\subsection{IPv4}
\begin{frame}
 \frametitle{Internet Protocol Version 4}
  \begin{block}{IPv4}
	  \begin{itemize}
	  	\item Definiert 1981 im RFC \footnote{Request for Comments} 791
	  	\item Beispiel Adresse: 8.8.8.8
  		\item nur $2^{32}$ (4294967296) = 4 Milliarden Adressen
	  \end{itemize}
  \end{block}
\end{frame}

\subsubsection{Verfügbare Adressen}
\begin{frame}  
  \frametitle{Internet Protocol Version 4}
  \framesubtitle{17.Dezember 2012 nur noch 16,84 Million Adressen verfügbar}
	\begin{figure}
	\includegraphics[scale=0.4]{IPv4_pool.png}
			\caption{RIPE \footnote{Réseaux IP Européens Network Coordination Centre} \\ \tiny{\textcolor{gray}{\url{http://www.ripe.net/internet-coordination/ipv4-exhaustion/ipv4-available-pool-graph}}}}
	\end{figure}
\end{frame}

\subsection{IPv6}
\begin{frame}
  \frametitle{Internet Protocol Version 6}
  \begin{block}{IPv6}
	  \begin{itemize}
  		\item Definiert 1998 im RFC \footnote{Request for Comments} 2460 
  		\item Beispiel Adresse: 2001:4860:4860::8888
		\item immer mehr Geräte sind über das Internet steuerbar
		\item immer mehr Smartphones und weitere mobile Geräte verfügbar
		\item intelligente Stromzähler \footnote{Smart Metering} benötigen ebenfalls IP Adressen
	  \end{itemize}
  \end{block}
\end{frame}


%\frame{\tableofcontents[currentsection]}

\section{Nachteile IPv6}
\begin{frame}
  \begin{alertblock}{Nachteile Internet Protocol Version 6}
	  \begin{itemize}
	  	\item Anpassungen auf IPv6 Fähigkeit verursacht Kosten
	  	\begin{itemize}
		    \item Infrastruktur
		    \item Bestriebssysteme 
		    \item Anwendungssoftware
		\end{itemize}	    
	    \item Mögliche Sicherheitsprobleme durch 
		\begin{itemize}
		     \item Wegfallen von IPv4 NAT \footnote{Network Address Translation} 
		     \item direkter Ende zu Ende Kommunikation
		\end{itemize}
	  \end{itemize}
  \end{alertblock}
\end{frame}

\section{Vorteile IPv6}
\begin{frame}
  \begin{block}{Vorteile Internet Protocol Version 6}
	  \begin{itemize}
  		\item $2^{128}$ (3,402823669 * $10^{38}$) = > 340 Sextillionen Adressen
  		\item Jedes Endgerät hat eine eindeutige IP-Adresse 
  		\item Wegfall von NAT \footnote{Network Address Translation}
  		\item Plug \& Play - Autokonfiguration SLAAC \footnote{Stateless Address Autoconfiguration} mit DAD \footnote{Duplicate Address Detection}
 		\item Effizienteres Routing
 		\item erweiterter Header
	  \end{itemize}
  \end{block}
\end{frame}



\section{Fazit und Ausblick}
\begin{frame}
 \frametitle{das Internet Protocol Version 6 kommt}
	Viele große Diensteanbieter (Google, Facebook etc.) nutzen schon seit längerer Zeit beide Protokolle im Dual Stack Betrieb. Der 06. Juni 2012 wurde zum World IPv6 Launch Day. An diesem Tag ging IPv6 offiziell in Betrieb. Der ISP (Deutsche Telekom AG) bindet mittlerweile auch Privatkunden, zusätzlich mit IPv6, an das Internet an.
	\begin{figure}
	\includegraphics[scale=0.4]{World_IPv6_launch_logo_128.png}
			\caption{IPv6 Logo \\ \tiny{\textcolor{gray}{\url{http://www.worldipv6launch.org}}}}
	\end{figure} 
\end{frame}

\section{Quellen}
\begin{frame}
 \frametitle{Quellen}
  \begin{block}{}
	  \begin{itemize}
  		\item IPv4 \url{http://tools.ietf.org/html/rfc791}
  		\item IPv6 \url{http://tools.ietf.org/html/rfc2460}
  		\item IPv6 SLAAC \footnote{Stateless Address Autoconfiguration} \url{http://tools.ietf.org/html/rfc4862}
  		\item http://www.worldipv6launch.org/
	  \end{itemize}
  \end{block}
\end{frame}

\subsection*{Ende}
\begin{frame}
	\begin{block}{}	
		\begin{center}
			Vielen Dank für Ihre Aufmerksamkeit. \\
			\Author{}
		\end{center}	
	\end{block}
\end{frame}

\end{document}


